% Template for PLoS
% Version 1.0 January 2009
%
% To compile to pdf, run:
% latex plos.template
% bibtex plos.template
% latex plos.template
% latex plos.template
% dvipdf plos.template
\DeclareFixedFont{\auacc}{OT1}{phv}{m}{n}{12}   % Patch by Gerry 3/21/07
\documentclass[11pt]{article}

% amsmath package, useful for mathematical formulas
\usepackage{amsmath}
% amssymb package, useful for mathematical symbols
\usepackage{amssymb}
\usepackage{wasysym}
\usepackage[table]{xcolor}
\usepackage{longtable}
\usepackage{lscape}
% graphicx package, useful for including eps and pdf graphics
% include graphics with the command \includegraphics
\usepackage{graphicx}  
\usepackage{verbatim}
\usepackage{float}
\floatstyle{boxed} 
\restylefloat{figure}
% citep package, to clean up citations in the main text. Do not remove.
%\usepackage{cite}
\usepackage{natbib}
\usepackage{lmodern}
\usepackage[T1]{fontenc} 
\usepackage{textcomp}
\usepackage{lineno} 
\newcommand{\carlos}[1]{\textcolor{red}{#1}}
%\newcommand{\co}[1]{{\color{red} #1}}   
% Use doublespacing - comment out for single spacing
%\usepackage{setspace} 
%\doublespacing

% Text layout
\topmargin 0.0cm
\oddsidemargin 0.5cm
\evensidemargin 0.5cm
\textwidth 17cm 
\textheight 21cm

% Bold the 'Figure #' in the caption and separate it with a period
% Captions will be left justified
%\usepackage[labelfont=bf,labelsep=period,justification=raggedright]{caption}

% Use the PLoS provided bibtex style

% Remove brackets from numbering in List of References
%\makeatletter
%\renewcommand{\@biblabel}[1]{\quad#1.}
%\makeatother

% Leave date blank


\pagestyle{myheadings}
%% ** EDIT HERE **
%* EDIT HERE **
%% PLEASE INCLUDE ALL MACROS BELOW

%% END MACROS SECTION

\begin{document}
\thispagestyle{empty}
\title{{\bf How to represent fluctuations at spatiotemporal scales?}}
\vspace{-0.1 in}
  %\author{Blake Matthews$^{1,*}$ and Carlos J. Meli\'an$^{2}$\\
  % $^1$Department of Aquatic Ecology,\\Swiss Federal Institute of
  %  Aquatic Science and Technology, Switzerland.\\    
  % $^2$Fish Ecology and Evolution Department,\\ Center for Ecology, Evolution and Biogeochemistry,\\
  %  Swiss Federal Institute of Aquatic Science and Technology, Switzerland.\\
  %  \\
  %  Keywords: ecological feedbacks, divergence, landscape genetics, speciation theory,\\
  %  gene flow, assortative mating, ecological drift, individual based model.\\
  %  Type of Article: X\\
  %  Number of figures: X; Number of tables: X\\
  %  $\ast$ Corresponding author: e-mail:blake.matthews@eawag.ch.}}

\date{}
\maketitle
\baselineskip=8.75 mm
%\linenumbers
%\modulolinenumbers[2]
\doublespacing
%\newpage
\markboth{{\small}

%\section*{Abstract} 
%  \newpage
\section*{Figures}

\begin{figure}[!ht]
\begin{center}
\includegraphics[width=5.5in]{scalegraphsml.eps}
\end{center}
\caption{This diagram illustrates the approximate spatial and temporal
  ranges for a variety of oceanic phenomena, along with estimated
  observational capabilities of various research platforms. Three
  basic trophic levels (phytoplankton, zooplankton, and fish) are
  shown. Thus, aircraft observations are ideal for large-scale, high
  resolution regional (synoptic) characterizations, encompassing a
  single tidal cycle and population variability at a scale of 10-1000
  meters. In contrast, satellites can best observe variability over
  10-1000 kilometers, with a maximum temporal resolution of about 1
  day. Ship observations can be at very high spatial resolution but
  are more limited in spatial range than aircraft or satellites.}
\end{figure}

\begin{figure}[!ht]
\begin{center}
\includegraphics[width=5.5in]{fig14.2Holling1992.eps}
\end{center}
\caption{Temporal and spatial scales within which large
  wading birds operate daily and over their lifetimes (modified from
  Holling, 1992).}
\end{figure}

\begin{figure}[ht!]
\begin{center}
\includegraphics[width=5.5in]{OSjDM.eps}
\end{center}
\caption{Oceanic processes for temporal scales spanning 10 orders of
  magnitude, while spatial scales span 12 orders of
  magnitude. Presumably mantle solvers would extend the upper bounds
  on each of these scales. Downloaded from
  http://earthscience.stackexchange.com/questions/49/similarities-between-grand-circulation-solvers-and-mantle-convection-solvers.}
\end{figure}

\begin{figure}[ht!]
\begin{center}
\includegraphics[width=5.5in]{atmosphere_scales.eps}
\end{center}
\caption{Atmospheric scales}
\end{figure}

\begin{figure}[ht!]
\begin{center}
\includegraphics[width=5.5in]{space_time_scale.eps}
\end{center}
\caption{Atmospheric scales again}
\end{figure}

%\newpage
%\bibliographystyle{ecography}
%\bibliography{EvolutionMultilevel}
\end{document}


