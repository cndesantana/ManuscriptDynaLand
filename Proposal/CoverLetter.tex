\documentclass{article}
\usepackage[utf8]{inputenc}

\pagenumbering{gobble}
\begin{document}
\parskip 6pt
\baselineskip 12pt
\noindent Dear Dr. Holyoak, dear members of the Editorial Board,\\
\\

Please find attached our manuscript entitled \emph{``Metacommunities in dynamic landscapes``} to be considered for publication as a \emph{Letter} article in \textbf{Ecology Letters}. In this manuscript we explore the effect of periodic fluctuations in landscape connectivity on species richness at both the temporal and the spatial scale. 

Predicting biodiversity dynamics in fast-changing landscapes is one of the most important problems in ecology and society. Most species extinctions are driven by landscape fragmentation but the effect of fluctuations in landscape connectivity on biodiversity dynamics has been poorly explored. The interplay between local patch dynamics and larger scale spatial processes is well documented in driving species richness and coexistence in metacommunities. However, landscape dynamics encompasses two major processes, i.e., patch dynamics and variation in landscape connectivity. Variation in landscape connectivity at all spatial and temporal scales may affect species richness and community structure. Yet, the effect of changing landscape connectivity together with migration dynamics on biodiversity dynamics remains poorly explored in metacommunity theory.

Predictions from experiments and field data have shown that high landscape connectivity promotes higher species richness than low connectivity but this pattern is not general because examples demonstrating high diversity in low connected landscapes also exist. Landscape connectivity shows intraday and daily to seasonal or larger time scale fluctuations but their role to understand the effect of landscape connectivity on species richness is lacking in metacommunity theory. Here, we connect the many factors that drive landscape connectivity by varying the amplitude and frequency determining the dispersal radius in spatial networks, showing that landscape connectivity plays a key role in predicting species richness. We explore the effect of periodic fluctuations in landscape connectivity on species richness at both the temporal and the spatial scale, varying both the amplitude and the frequency of changes in the dispersal radius of several orders of magnitude. We use highly replicated numerical simulations to check theoretical predictions.

We show that the fluctuations of landscape connectivity support metacommunities with higher species richness than static landscapes in fragmented landscapes. We show that the variance of regional species richness peaks with an intermediate number of isolated components in dynamic landscapes suggesting high regional species richness can also occur in dynamic landscapes with highly fragmented landscapes. Our results also show a dispersal radius threshold below which species richness drops dramatically in static landscapes. This threshold is not observed in dynamic landscapes for a broad range of values of amplitude and frequency determining landscape connectivity. Our analysis suggest that landscape connectivity together with patch dynamics can provide new testable predictions about species diversity at local and regional scales in fast-changing landscapes.

Our work will be of interest to experimentalists, managers and theoreticians working at the interface of biodiversity, conservation and landscape dynamics at short and large spatiotemporal scales. We believe our work also represents an important advance to understand metacommunity dynamics in fragmented and fast-changing landscapes.  Our results suggest that changes in biodiversity inferred only from static landscapes or from landscapes with only patch dynamics are likely to be improved taking into account landscape connectivity. Moreover, given the rapid change, and loss, that habitats are facing, there is an urgent need of new theory enabling reliable predictions of the consequences of such changes on biodiversity.

Because of their contribution to and knowledge of this subject, we suggest the following scientists as potential editors and reviewers:

Potential editors are:

\begin{itemize}
\item Marcel Holyoak
\item Sergio Navarrete
\end{itemize}

Possible reviewers are:

\begin{itemize}
\item Otso Ovaskainen
\item Juan E. Keymer
\item Joel Rybicki
\end{itemize}

All authors have agreed to the submission of the manuscript and all persons entitled to authorship have been named. The work has not been submitted to any other journal and represents our original research. We would like to mention that we have sent a proposal for the submission of this manuscript to \emph{Ideas and Perspectives} section in \textbf{Ecology Letters}. Such proposal was rejected, however the Editor-in-Chief Professor Marcel Holyoak has encouraged us to submit this ms. as a Letter.

We thank you in advance for the attention dedicated to our manuscript

Yours sincerely,
\vspace{0.1 in}

\noindent Charles de Santana (corresponding author: charles.desantana@eawag.ch)\\
Jan Klecka\\
Gian Marco Palamara\\
Carlos J. Meli\'an\\
\end{document}

